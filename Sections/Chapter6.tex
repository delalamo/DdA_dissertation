\clearpage % clear the prior chapter's page

\chapter{Perspectives and future directions}\label{ch:conclusions}
%\vspace{-7mm}
%\bigskip

\section{Synopsis of experimental findings}

% Talk about dynamics in the LeuT-fold
% Talk about advances in modeling using EPR data
% Talk about more general advances in protein modeling

This document presents experimental research into the pH-dependent activation mechanism and conformational dynamics of GadC, a "virtual proton pump" found in the \gls{apc} transporter family. Additionally, it describes a series of structural models generated using these experimental data that attempt to recapitulate the conformation adopted by GadC in solution. While the structural dynamics of more distant homologs, such as prokaryotic and eukaryotic \gls{nss}s and \gls{sss}s, continue to be extensively studied, prior to this research only a single investigation in a member of the \gls{apc} family, the homologous serine/threonine antiporter SteT, had been conducted. It is noteworthy that although the work discussed in this dissertation coincided with several breakthroughs in membrane protein structural biology (discussed below), our body of knowledge with respect to structural dynamics of LeuT-fold antiporters barely grew. High-resolution structures of eukaryotic exchangers with disease relevance, including Lat1 and Lat2, were unaccompanied by detailed descriptions of how or even whether they isomerize when bound to ligands.

Given the sensitivity of structural homologs LeuT, Mhp1, and vSGLT to ligands, it came as a surprise that no ligand-dependent conformational dynamics were observed in GadC. Sodium-coupled symporters rely on ion gradients to drive the energetically unfavorable uptake of amino acids and other nutrients into the cell. By contrast, in the low-pH conditions under which GadC is hypothesized to be active, depletion and buildup of intracellular glutamate and \gls{gaba}, respectively, lead both substrates to be transported down their concentration gradients. The environmental concentrations of both substrates on either side of the cell may thus be sufficient to enforce productive substrate movement required by the bacterial cell under low-pH conditions. Thus, the data presented in Chapter \ref{ch:gadc} and Appendix \ref{app:gadc_supp} of this dissertation reinforce the structural and conformational diversity of transporters with the LeuT-fold, further highlighting the extent to which the energy landscapes of proteins with the same fold can diverge. A key takeaway from this work is that experimental evidence of ligand-dependent conformational dynamics in one transporter may not fully correspond with those of structural homologs; in other words, conservation of transporter dynamics at the family-level may not be guaranteed. An important implication of this conclusion is that it may suggest that the conformational dynamics of transporters in eukaryotic organisms in general and humans in particular may not match those of bacterial model systems (see section \ref{sec:leutintro_homologs}). For this reason, it remains unclear whether these results extend to human amino acid exchangers with disease relevance such as Lat1 \citep*{Lee2019, Yan2019} or xCT \citep*{Oda2020}. Ultimately this hypothesis will be tested as investigations into human proteins become more widespread.

\subsection{Perspectives on the effect of substrates on the conformational dynamics of GadC}

The \gls{deer} data presented here suggests that neither substrate biases the conformational dynamics of GadC, a finding which led us to postulate that glutamate and \gls{gaba} affect the protein's kinetics, rather than its thermodynamics - such a mechanism would almost certainly be missed by measurements carried out using the \gls{deer} technique. Under the proposed mechanism, which is outlined in section \ref{sec:gadc_discussion}, the contribution of substrate binding to conformational dynamics is limited to stabilization of a hypothetical high-energy transition state that is not traversable under apo conditions. This model, although simple, would explain both the antiport mechanism forbidden substrate-free isomerization as well as the absence of any substrate-mediated changes in the conformational dynamics of GadC observed using \gls{deer}. Nevertheless, as the \gls{deer} technique only interrogates protein thermodynamics, rather than kinetics, other experiments would be required to test this hypothesis. Direct evaluation of changes in protein kinetics can be achieved using \gls{hdxms} \citep*{Oganesyan2018}, single-molecule \gls{fret} \citep*{Schuler2013}, or fluorine \gls{nmr} \citep*{Manglik2015}.

\subsection{Perspectives on the pH-dependent activation mechanism of GadC}

This directly ties into outstanding questions regarding the mechanism of pH-dependent substrate transport. Data collected in radioligand transport assays show how activity spikes at low pH and does not appear to plateau in the pH range under experimental observation (Figure \ref{fig:gadc_main_transport}.B and C). By contrast, detachment of the C-terminus appears to occur with a pKa of 6.0 to 6.25 (Figure \ref{fig:gadc_main_tail}.B, C, and D), which likely rules out the contribution of this domain to the increase in transport rate observed at pH 4.0-5.0. In fact, the change in glutamate exchange observed at pH 6.0-6.5 is negligible compared to the high rates of transport observed at lower pHs, which calls into question the role of this domain in regulating transport at neutral pH.

Two questions naturally follow this line of thinking. First, what other mechanism could explain the pH-dependent activity observed in GadC? Although protonation of the substrates' $\mathrm{\upgamma}$-carboxylate (pKa: 4.25), a prerequisite for transport, may partially explain this phenomenon, abrogated transport of glutamine, which mimics protonated glutamate, at neutral and alkaline pHs is inconsistent with this hypothesis \citep*{Ma2013}. In the homologous arginine/agmatine antiporter AdiC, activation has been attributed to the proton sensing residue tyrosine Y74, located on the intracellular amphipathic helix connecting \gls{tmh}s 2 and 3 \citep*{Wang2014}. Whereas the wildtype similarly undergoes inactivation at neutral pH, AdiC-Y74A maintained high transport activity regardless of pH. This research is relevant because AdiC-Y74F maintained the same pH-dependent inactivation profile of the WT, and in GadC a phenylalanine is found at the equivalent position (residue 76). However, we note that spin-labeled cysteine mutants at residue 77 showed little to no change in either \gls{deer} distance measurements or \gls{cw} spectra, and was inactive at neutral pH. Nevertheless, this hypothesis can be directly tested on F76A background mutants using radioligand transport assays such as those outlined in section \ref{sec:gadc_transport_assay}. A second hypothesis in AdiC, proposed following \gls{md} simulations, suggests that protonation of E218 drives dissociation of the substrate from the active site \citep*{Zomot2011}. This residue is strictly conserved in the pH-dependent "virtual proton pumps" but not the neutral-pH homologous transporters such as ApcT and Lat1 \citep*{Ma2012}. A simple test of this hypothesis would be to measure the dissociation constants of radiolabeled glutamate, \gls{gaba}, and glutamine as a function of pH using a scintillation proximity assay \citep*{Quick2007}, which measures substrate binding affinity in detergent-solubilized transporters, in both wildtype and E218Q mutants of GadC. The role of this proposed proton would be equivalent to that of potassium in LeuT, which serves to displace sodium from the substrate-binding site but is otherwise uninvolved in the transport cycle \citep*{Billesbølle2016}.

Second, if the C-terminal domain is negligibly involved in regulation of pH-dependent transport, then what is its primary purpose? We speculate that this domain binds the glutamate decarboxylase GadB, which is cotranscribed with GadC and has been shown to partition to the membrane fraction via an unknown mechanism at acidic pH, but not at neutral pH. In fact, the 2003 publication presenting the crystal structure of GadB proposed this exact hypothesis in passing \citep*{Capitani2003}. While the structure of GadC, determined and published a decade later, was entirely consistent with this mechanism \citep*{Ma2012}, no experimental evidence supporting or refuting this possible protein-protein interaction has, to our knowledge, been published. Moreover, since the publication of GadC's structure, an equivalent mechanism was observed and demonstrated in the structural homolog DrSLC38A9, which has a similar N-terminal domain embedded in its intracellular cavity that, when released into the cytoplasm, binds and recruits the regulatory complex mTORC1 to the lysosomal membrane. A pH-dependent GadB/GadC interaction could easily be tested by spin labeling the C-terminal domain and observing mobility changes in the \gls{cw} spectrum as a function of both pH and GadB concentration (see Figure \ref{fig:gadc_main_tail}.D for an example of this experiment), as slower tumbling times would be expected following binding of a \SI{330}{kDa} soluble protein. Follow-up experiments include visualization of GFP-labeled GadB using fluorescence microscopy \emph{in vivo} in cells expressing either full-length or truncated GadC at neutral and acidic pH. Alternatively, the structural basis of this interaction can be determined by crystallography of GadB at low pH with a peptide fragment whose sequence matches that of the C-terminal domain of GadC.

\subsection{Perspectives on the IF-occluded conformation observed using DEER}

We now turn our attention to the \gls{if}-occluded conformation modeled using the experimental \gls{deer} data. Under physiological conditions, antiporters belonging to the \gls{apc} transporter family exchange substrates present at micromolar concentrations outside the cell, but millimolar concentrations inside the cell. In some proteins, such as the alanine/serine/cysteine antiporter BasC, this leads to apparent Km values in the micromolar range during import (\gls{of}-to-\gls{if}) but in the millimolar range during export (\gls{if}-to-\gls{of}). Technical limitations prevented the \gls{deer} experiments presented in this dissertation from being carried out in a gradient, leading to substrate concentrations identical on both sides of the membrane. We therefore speculate that if GadC interacts with substrates with a similar sidedness as BasC, then under the experimental conditions discussed in this dissertation, this would be expected to lead to more \gls{of}-to-\gls{if} isomerization than \gls{if}-to-\gls{of}, consistent with a preponderance of \gls{if} GadC observed in our experiments. Stabilization of \gls{of} GadC may be achieved by introducing a gradient, which would require reconstitution of GadC into proteoliposomes.  To address the possibility that not all GadC molecules are correctly oriented in the proteoliposome, the membrane-impermeable reducing agent TCEP may need to be introduced following reconstitution to reduce any spin-labeled cysteine residues on the wrong side of the membrane (e.g. intracellular cysteine residues on the outside of the proteoliposome).

Other conformations in the transport cycle could conceivably be visualized using \gls{deer} by introducing a pH gradient. Maintenance of a pH gradient for extended periods of time has previously been shown to require specific lipid profiles; previous experiments in AdiC and GadC have relied on liposomes comprised of 3:1 POPE:POPG to maintain an outer pH of 2.2 and an inner pH of 5.0 \citep*{Tsai2012, Tsai2013, Tsai2013a}. When executed in conjunction with the preparatory steps outlined above, this experiment could reveal how pH gradients contribute to stabilization of \gls{of} GadC and sampling of discrete conformational intermediates in the presence or absence of substrates.

To summarize the experimental findings, we found that 1) the structural basis of pH-dependent activation is partially, but not fully, mediated by detachment of the C-terminal domain as previously hypothesized \citep*{Ma2012}, and 2) the transporter predominantly adopts an inward-facing occluded conformation regardless of whether substrates are presence or not. These findings advance our understanding of transporters with the LeuT-fold by reinforcing the divergent energy landscapes underpinning function. Further research into the breadth of transport mechanisms mediated by symporters, antiporters, and permeases will be necessary to determine whether the observations made in GadC are restricted to amino acid exchangers or transporters in the APC family.

\section{Synopsis of methodological advancements}

While this dissertation nominally focused on studies of GadC, the overwhelming majority of the results presented focus on the development of methods for modeling protein structures using sparse experimental data, particularly \gls{deer} data. These methods were designed to tackle the sparse and imprecise nature of the data being collected. Thus, in contrast with equivalent methods designed to model proteins using \gls{cryoem} density or residue coevolutionary restraints, acknowledging the uncertainty and uneven distribution inherent to the experimental data presented in Chapter \ref{ch:gadc} was fundamental to minimizing the risk of overfitting, which could lead to spurious conclusions (results obtained in \emph{de novo} folding benchmarks of Bax and ExoU in Chapter \ref{ch:rosettadeer} illustrate how incorrectly folded models can satisfy the experimental data, which is exactly the outcome to be avoided). Therefore, the work presented in this dissertation took a multi-pronged approach to maximize the contribution of these data during modeling:

\begin{itemize}
    \item Uncertainty was minimized by explicitly modeling the spin label ensemble (Chapter \ref{ch:rosettadeer}). Compared to the previous implementation of \gls{deer} restraints in Rosetta, the \gls{cone} model, improvements in modeling precision were observed in every protein.
    \item An effective scoring function was then determined by comparing several candidate functions in Appendix \ref{app:scoring}. This included the possibility that multiple conformers were present in the data, which the distributions in Chapter \ref{ch:gadc} could not rule out.
    \item Further improvements in modeling precision were possible in cases where the starting structure was consistent with a set of \gls{deer} data. Multilateration of the spin labels using the algorithm discussed in Chapter \ref{ch:multilateration} could conceivably be used to more precisely determine the conformation of interest. Unfortunately, GadC did not adopt the conformation observed in the crystal structure, precluding the use of this method.
    \item Effective sampling methods discussed in Appendix \ref{app:confchangemover} allowed modeling to be focused on the immediate conformational vicinity of the starting structure, which prevented precious computational resources from being wasted on sampling unrealistic conformers.
    \item Finally, a statistical potential capturing our expectation that the structure of GadC resembles those of its homologs served as an additional source of regularization that prevented outrageous structures from being considered (see section \ref{sec:gadc_potential}).
\end{itemize}

The combined approach allowed models of \gls{if}-occluded GadC to be modeled using only 23 experimental restraints. The resulting models at both pHs closely resembled the structures of two homologs, ApcT and GkApcT, as well as a model generated using RoseTTAFold (see section \ref{sec:conclusion_integrative_modeling} below and Figure \ref{fig:gadc_supp_rosettafold}). Nevertheless, uncertainty in modeling prevented differences between conformations generated using low- or neutral-pH data from being observed. While that may hint at additional computational innovations that have yet to be realized, it may also suggest that imprecise and/or sparse experimental measurements can only go so far in resolving small-scale conformational changes. Mitigation or elimination of experimental uncertainty could be achieved using more rigid spin labels, such as bifunctional labels or imidazole-derived spin labels, which sample fewer conformers.

In summary, the computational work presented here describes a means to directly integrate \gls{deer} data for protein modeling. The results we obtained when using the raw data as experimental restraints, rather than as a means of checking the correctness of structures after the fact, suggests that sparse experimental \gls{deer} data can be integrated with computational modeling to complement a wide variety of tasks. In this dissertation the use of \gls{deer} data is limited to \emph{de novo} protein fold prediction (Chapter \ref{ch:rosettadeer}, homology modeling (Chapter \ref{ch:gadc}), and conformational change modeling (Chapters \ref{ch:multilateration} and \ref{ch:gadc}), but the data can conceivably be adapted to achieve other tasks, such as to predict the conformations of flexible loop regions that might be unresolved in experimental structures (see Appendix \ref{app:loophash}), or determine the location of paramagnetic ligands (see reference \citep*{Gaffney2012}). However, integrating \gls{deer} data with other forms of experimentally collected information, such as SAXS data or low-resolution \gls{cryoem} density, will require further fine-tuning.

\section{Final thoughts: perspectives on integrative modeling using sparse data}\label{sec:conclusion_integrative_modeling}

As was discussed in section \ref{sec:deerintro_general_integrative}, the overarching goal of integrative modeling is to either explain data that has already been collected or predict future observations \citep*{Hofman2021}. While the former was the predominant goal of the methods development projects discussed here, both approaches contributed to the research presented in Chapter \ref{ch:gadc} of this dissertation. Although not discussed, experimental design of spin label pairs in GadC was initially guided by an \gls{of} homology model of GadC generated using RosettaCM \citep*{Song2013} with various structures of AdiC serving as templates \citep*{Fang2009, Gao2009}. A challenge when using homology models to predict conformational changes between states with \gls{rmsd} values of \SIrange{3}{5}{\angstrom} is that physiologically relevant structural movements are difficult to distinguish from modeling artifacts. By contrast, pairs of experimental structures of a single homolog in both \gls{of} and \gls{if} conformations can reveal more precisely which regions of a protein move and which stay fixed. Unfortunately, structural characterization of an \gls{apc} transporter in both conformations did not occur until March 2021, after data collection was concluded. Comparison of these two structures, shown in Figures \ref{fig:leutintro_aa} and \ref{fig:leutintro_rmsf}, reveal movement in virtually every helix in the core LeuT-fold transmembrane domain. Importantly, they showed helical movements that were initially predicted by the \gls{of} homology model of GadC, but misconstrued as modeling artifacts, particularly in the hash domain and \gls{tmh}10. As a result, no spin pairs were designed to specifically interrogate these movements.

The shortcomings in using integrative modeling to design experiments of this nature hint at a larger problem regarding the use of \gls{epr} spectroscopy as an exploratory tool for structural studies. Fortunately, this research coincided with two methodological advancements with major implications for the field of integrative structural biology. First, steady improvements in both software and hardware allowed single-particle \gls{cryoem} to mature from a technique capable of viewing the topologies of large proteins and complexes (>\SI{150}{kDa}) at low-to-medium resolution to one capable of resolving the structures and dynamics of even medium-sized proteins, such as \gls{sert} (\SI{70}{kDa}), to atomic detail \citep*{Coleman2019, Zhong2021}. Second, state-of-the-art \emph{de novo} protein structure prediction algorithms were recently developed that are capable of routinely achieving sub-angstrom modeling accuracy from sequence alone \citep*{Baek2021, Jumper2021}. Appendix \ref{app:alphafold2} presents anecdotal evidence that multiple discrete conformers may be modeled to high accuracy this way, solving a major outstanding problem in \emph{de novo} structure prediction \citep*{Nicoludis2018}. Therefore, access to atomic-detail structures and structural models is expected to be less of a barrier to high-impact structural biology research in the coming years.

What will be the effect of these developments on integrative modeling? The research outlined in this dissertation followed a playbook used in previous integrative structural biology investigations \citep*{Kazmier2014a, Paz2018} that focuses on two types of models: structural models and mechanistic models. These technological advancements point to a future in which the integration of sparse data, such as distance data collected using \gls{epr}, for the purposes of structure prediction may soon be unnecessary. By contrast, these technological improvements have the potential to facilitate the design of enormously informative experiments seeking to describe how and when different structures of a protein, either obtained experimentally or predicted \emph{de novo}, interconvert in response to mutagenesis, ligand binding, or environmental changes such as pH or lipid composition. As stated above, it was unclear during data collection if the absence of conformational dynamics in the \gls{deer} data resulted from structural uniformity or from uninformative spin pair design. Having multiple protein structures and/or high-accuracy structural models in different conformations can mitigate the possibility of the latter, which would be a welcome change when designing experiments that reports on the characteristics of a protein's energy landscape. Thus, experimental design that aims to track population changes, rather than precise conformational details, plays to the strength of the \gls{deer} technique and will ensure its relevance for years to come.